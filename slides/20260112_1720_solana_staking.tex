\documentclass[8pt,aspectratio=169]{beamer}
\usetheme{Madrid}
\usecolortheme{default}

% Packages
\usepackage{graphicx}
\usepackage{amsmath,amssymb}
\usepackage{booktabs}
\usepackage{tikz}
\usepackage{hyperref}

% Colors (Solana-inspired)
\definecolor{solpurple}{HTML}{9945FF}
\definecolor{solgreen}{HTML}{14F195}
\definecolor{darkbg}{HTML}{0D1117}
\definecolor{accentblue}{HTML}{58A6FF}

\setbeamercolor{title}{fg=solpurple}
\setbeamercolor{frametitle}{fg=solpurple}
\setbeamercolor{structure}{fg=solpurple}
\setbeamercolor{block title}{bg=solpurple,fg=white}

% Custom commands
\newcommand{\bottomnote}[1]{\vfill\small\textit{#1}}
\newcommand{\highlight}[1]{\textcolor{solpurple}{\textbf{#1}}}

% Title
\title{SOL Staking \& Capital Structure Optimization}
\subtitle{A Framework for Permanent SOL Accumulation}
\author{Digital Finance Research}
\institute{github.com/Digital-AI-Finance/solana-staking}
\date{January 2026}

\begin{document}

% Title slide
\begin{frame}
\titlepage
\end{frame}

% Table of Contents
\begin{frame}{Agenda}
\tableofcontents
\end{frame}

%=============================================================================
\section{Solana Staking Fundamentals}
%=============================================================================

\begin{frame}[t]{Why Solana for Treasury Strategy?}
\textbf{Key Advantages}
\begin{itemize}
    \item Native staking yield: \highlight{7-9\% APY} without smart contract risk
    \item High throughput: 65,000+ TPS enables real economic activity
    \item Growing DeFi ecosystem: \$10B+ TVL in liquid staking
    \item Institutional adoption accelerating (ETF products emerging)
\end{itemize}
\vspace{0.3cm}
\textbf{Staking Economics}
\begin{itemize}
    \item Current staking ratio: \highlight{68.57\%} of supply staked
    \item Inflation-funded rewards distributed to stakers
    \item MEV revenue adds additional yield via Jito
\end{itemize}
\bottomnote{Source: Solana Beach, January 2026}
\end{frame}

\begin{frame}[t]{Solana Inflation Schedule}
\begin{center}
\includegraphics[width=0.65\textwidth]{../charts/01_staking_yield_history/chart.pdf}
\end{center}
\bottomnote{Inflation decreases 15\% annually until reaching 1.5\% terminal rate around 2032.}
\end{frame}

\begin{frame}[t]{Inflation \& Staking APY Mathematics}
\textbf{Inflation Rate Over Time}
\begin{equation}
    i(t) = \max\left(i_0 \cdot (1-d)^t, i_{\text{terminal}}\right)
\end{equation}
Where: $i_0 = 8\%$, $d = 15\%$, $i_{\text{terminal}} = 1.5\%$

\vspace{0.3cm}
\textbf{Staking APY Approximation}
\begin{equation}
    APY_{\text{staking}} \approx \frac{i(t)}{s} \cdot (1 - f_{\text{validator}})
\end{equation}
Where: $s$ = staking ratio, $f$ = validator commission

\vspace{0.3cm}
\textbf{Current Values (2025)}
\begin{itemize}
    \item Inflation: $i(4) = 8\% \times 0.85^4 \approx 4.07\%$
    \item Staking APY: $4.07\% / 0.68 \approx 6.0\%$ base + MEV
\end{itemize}
\bottomnote{Real yields vary by validator performance and MEV participation.}
\end{frame}

\begin{frame}[t]{Liquid Staking Token (LST) Landscape}
\begin{center}
\includegraphics[width=0.65\textwidth]{../charts/01b_lst_market_share/chart.pdf}
\end{center}
\bottomnote{Jito leads with 44\% market share due to MEV integration.}
\end{frame}

\begin{frame}[t]{Native vs Liquid Staking Trade-offs}
\begin{columns}
\begin{column}{0.48\textwidth}
\textbf{Native Staking}
\begin{itemize}
    \item Direct delegation to validators
    \item No smart contract risk
    \item Withdrawal: 2-3 day unstaking
    \item Full control over validator selection
\end{itemize}
\end{column}
\begin{column}{0.48\textwidth}
\textbf{Liquid Staking (LSTs)}
\begin{itemize}
    \item Receive tradeable token (JitoSOL, mSOL)
    \item DeFi composability (lending, LP)
    \item Instant liquidity (swap vs SOL)
    \item Smart contract risk exposure
\end{itemize}
\end{column}
\end{columns}
\vspace{0.3cm}
\textbf{Recommendation}: Institutional treasuries should use \highlight{native staking} for core holdings, LSTs for DeFi yield enhancement on margins.
\bottomnote{LST smart contract audits are critical for risk assessment.}
\end{frame}

%=============================================================================
\section{Validator Business Model}
%=============================================================================

\begin{frame}[t]{Validator Economics Overview}
\textbf{Revenue Sources}
\begin{itemize}
    \item \highlight{Self-stake rewards}: Full APY on validator's own stake
    \item \highlight{Commission}: Percentage of delegator rewards (typically 5-10\%)
    \item \highlight{MEV tips}: Additional revenue via Jito bundle auctions
\end{itemize}
\vspace{0.3cm}
\textbf{Cost Structure}
\begin{itemize}
    \item Vote costs: $\sim$1.1 SOL/day ($\sim$400 SOL/year)
    \item Infrastructure: \$700-1,000/month (hardware, colo, bandwidth)
    \item Initial setup: \$8,000-15,000 one-time
\end{itemize}
\bottomnote{Profitable validators typically need 100K+ SOL in total stake.}
\end{frame}

\begin{frame}[t]{Validator Cost Structure}
\begin{center}
\includegraphics[width=0.65\textwidth]{../charts/02_validator_setup_costs/chart.pdf}
\end{center}
\bottomnote{Monthly operating costs total $\sim$\$1,100 plus 33 SOL vote costs.}
\end{frame}

\begin{frame}[t]{Validator Revenue Model}
\textbf{Annual Revenue Calculation}
\begin{equation}
    R = S_{\text{self}} \cdot APY + S_{\text{delegated}} \cdot APY \cdot c + MEV
\end{equation}
Where:
\begin{itemize}
    \item $S_{\text{self}}$: Self-stake (validator's own SOL)
    \item $S_{\text{delegated}}$: Delegated stake from others
    \item $c$: Commission rate (e.g., 5\%)
    \item $MEV$: Maximal Extractable Value tips (Jito)
\end{itemize}
\vspace{0.3cm}
\textbf{Example}: 500K total stake, 50K self-stake, 5\% commission, 7.5\% APY
\begin{align}
    R &= 50,000 \times 0.075 + 450,000 \times 0.075 \times 0.05 + MEV \\
    &= 3,750 + 1,687.5 + \sim 500 = \highlight{5,937.5 \text{ SOL/year}}
\end{align}
\bottomnote{At \$150/SOL, this equals $\sim$\$890K annual revenue.}
\end{frame}

\begin{frame}[t]{MEV Revenue Distribution}
\begin{center}
\includegraphics[width=0.65\textwidth]{../charts/03_mev_revenue/chart.pdf}
\end{center}
\bottomnote{MEV scales with stake weight; Jito-enabled validators earn additional tips.}
\end{frame}

\begin{frame}[t]{Breakeven Analysis}
\begin{center}
\includegraphics[width=0.65\textwidth]{../charts/02d_breakeven_stake/chart.pdf}
\end{center}
\bottomnote{Higher commission rates lower breakeven stake but may reduce delegations.}
\end{frame}

\begin{frame}[t]{Optimal Commission Rate}
\textbf{The Commission Trade-off}
\begin{itemize}
    \item Higher commission $\rightarrow$ More revenue per delegator
    \item Higher commission $\rightarrow$ Fewer delegators attracted
\end{itemize}
\vspace{0.3cm}
\textbf{Optimization Problem}
\begin{equation}
    \max_{c} \Pi(c) = D(c) \cdot APY \cdot c + S_{\text{self}} \cdot APY - \text{Costs}
\end{equation}
Where $D(c)$ is delegation demand as a function of commission rate.

\vspace{0.3cm}
\textbf{Market Observation}
\begin{itemize}
    \item Most competitive validators: 0-2\% commission
    \item Mid-tier validators: 5-8\% commission
    \item Superminority threshold affects delegation flows
\end{itemize}
\bottomnote{Commission optimization depends on validator reputation and performance.}
\end{frame}

%=============================================================================
\section{Corporate Finance for SOL Accumulation}
%=============================================================================

\begin{frame}[t]{The MicroStrategy Playbook}
\textbf{MicroStrategy's Bitcoin Treasury Strategy}
\begin{itemize}
    \item Began August 2020 with \$250M BTC purchase
    \item Issued \$3B+ in convertible bonds (0\% coupon, 35\% premium)
    \item Holdings: 660,624 BTC (\$62B as of Dec 2025)
    \item Stock appreciation: 1,204\% since first purchase
\end{itemize}
\vspace{0.3cm}
\textbf{Adapting for SOL}
\begin{itemize}
    \item Same capital structure approach: convertibles + equity
    \item \highlight{Added benefit}: Native staking yield (BTC has none)
    \item \highlight{Added benefit}: Validator business as revenue source
    \item Risk consideration: Higher volatility than BTC
\end{itemize}
\bottomnote{Source: Strategy Inc. investor presentations, SEC filings.}
\end{frame}

\begin{frame}[t]{Convertible Bond Structure}
\textbf{Zero-Coupon Convertible Mechanics}
\begin{itemize}
    \item Face value: \$1,000 per bond
    \item Coupon: 0\% (no periodic interest)
    \item Conversion price: Set at premium to current price (e.g., 35\%)
    \item Maturity: 5 years typical
\end{itemize}
\vspace{0.3cm}
\textbf{Value Decomposition}
\begin{equation}
    V_{\text{CB}} = V_{\text{bond}} + V_{\text{call}}
\end{equation}
\begin{itemize}
    \item Bond floor: PV of face value at risk-adjusted rate
    \item Embedded call: Black-Scholes option on underlying SOL
\end{itemize}
\bottomnote{Convertibles allow cheap capital (0\% interest) in exchange for upside participation.}
\end{frame}

\begin{frame}[t]{Convertible Payoff Diagram}
\begin{center}
\includegraphics[width=0.65\textwidth]{../charts/04_convertible_payoff/chart.pdf}
\end{center}
\bottomnote{At maturity, holder receives max(face value, conversion value).}
\end{frame}

\begin{frame}[t]{Black-Scholes Valuation}
\textbf{Embedded Call Option Pricing}
\begin{equation}
    C = S_0 N(d_1) - K e^{-rT} N(d_2)
\end{equation}
\begin{equation}
    d_1 = \frac{\ln(S_0/K) + (r + \sigma^2/2)T}{\sigma\sqrt{T}}, \quad d_2 = d_1 - \sigma\sqrt{T}
\end{equation}

\textbf{Example}: SOL = \$150, Strike = \$200, T = 5yr, $\sigma$ = 80\%, r = 4.5\%
\begin{itemize}
    \item $d_1 = \frac{\ln(150/200) + (0.045 + 0.32)5}{0.8\sqrt{5}} = 0.69$
    \item Call value per SOL: \$85.23
    \item Conversion ratio: 5 SOL per \$1,000 bond
    \item Option value per bond: \$426
\end{itemize}
\bottomnote{High SOL volatility makes convertibles attractive to arbitrage funds.}
\end{frame}

\begin{frame}[t]{PIPE Structure}
\textbf{Private Investment in Public Equity}
\begin{itemize}
    \item Private placement to institutional investors
    \item Typically 5-15\% discount to market price
    \item Registration rights for resale
    \item Lock-up periods (90-180 days)
    \item Often includes warrant sweeteners
\end{itemize}
\vspace{0.3cm}
\textbf{Advantages for SOL Treasury}
\begin{itemize}
    \item Faster execution than public offering
    \item Lower regulatory burden
    \item Targeted investor base (crypto-friendly institutions)
    \item Flexible terms negotiation
\end{itemize}
\bottomnote{PIPEs complement convertibles for capital raising flexibility.}
\end{frame}

%=============================================================================
\section{Capital Structure Optimization}
%=============================================================================

\begin{frame}[t]{Modigliani-Miller Framework}
\textbf{Value of Levered Firm}
\begin{equation}
    V_L = V_U + \text{PV(Tax Shield)} - \text{PV(Distress Costs)}
\end{equation}

\textbf{Trade-off Theory}
\begin{itemize}
    \item Tax shield: Interest payments are tax-deductible
    \item Distress costs: Bankruptcy risk increases with leverage
    \item Optimal leverage: Where marginal benefit = marginal cost
\end{itemize}
\vspace{0.3cm}
\textbf{Crypto Modification}
\begin{itemize}
    \item Higher volatility $\rightarrow$ Higher distress probability
    \item Staking yield $\rightarrow$ Additional interest coverage
    \item No dividends $\rightarrow$ Tax shield more valuable
\end{itemize}
\bottomnote{Traditional MM assumptions need adjustment for crypto treasury.}
\end{frame}

\begin{frame}[t]{Capital Structure Optimization}
\begin{center}
\includegraphics[width=0.65\textwidth]{../charts/06_capital_structure_frontier/chart.pdf}
\end{center}
\bottomnote{Optimal D/E ratio minimizes WACC while maintaining interest coverage.}
\end{frame}

\begin{frame}[t]{WACC Calculation}
\textbf{Weighted Average Cost of Capital}
\begin{equation}
    WACC = \frac{E}{V} \cdot r_e + \frac{D}{V} \cdot r_d \cdot (1-\tau)
\end{equation}

\textbf{Cost of Equity (MM Proposition II)}
\begin{equation}
    r_e = r_u + (r_u - r_d) \cdot \frac{D}{E} \cdot (1-\tau)
\end{equation}

\textbf{Example}: \$50M equity, 1.5x D/E, $r_u=20\%$, $r_d=6\%$, $\tau=21\%$
\begin{itemize}
    \item $r_e = 20\% + (20\%-6\%) \times 1.5 \times 0.79 = 36.6\%$
    \item $WACC = 0.4 \times 36.6\% + 0.6 \times 6\% \times 0.79 = 17.5\%$
\end{itemize}
\bottomnote{SOL staking yield (7.5\%) provides natural interest coverage.}
\end{frame}

\begin{frame}[t]{Interest Coverage Analysis}
\textbf{Interest Coverage Ratio}
\begin{equation}
    ICR = \frac{\text{Staking Income} + \text{MEV Revenue}}{\text{Interest Expense}}
\end{equation}

\textbf{Example Calculation}
\begin{itemize}
    \item Debt: \$75M at 6\% = \$4.5M annual interest
    \item SOL holdings: 833K SOL (at \$150)
    \item Staking income: 833K $\times$ 7.5\% $\times$ \$150 = \$9.37M
    \item ICR = \$9.37M / \$4.5M = \highlight{2.08x}
\end{itemize}
\vspace{0.3cm}
\textbf{Minimum Threshold}: ICR $>$ 1.5x recommended for stability
\bottomnote{Staking yield provides natural debt service coverage unique to SOL.}
\end{frame}

%=============================================================================
\section{Options Strategies for Yield Enhancement}
%=============================================================================

\begin{frame}[t]{Options Overlay Strategy}
\textbf{Enhancing Returns with Derivatives}
\begin{itemize}
    \item \highlight{Covered Calls}: Sell upside for premium income
    \item \highlight{Cash-Secured Puts}: Accumulate SOL at lower prices
    \item \highlight{Collars}: Downside protection with premium offset
    \item \highlight{Volatility Harvesting}: Exploit IV vs RV spread
\end{itemize}
\vspace{0.3cm}
\textbf{SOL Options Market}
\begin{itemize}
    \item Primary venues: Deribit, OKX
    \item Typical IV: 70-100\% annualized
    \item Liquidity: Concentrated in monthly expirations
    \item Strike availability: Limited compared to BTC/ETH
\end{itemize}
\bottomnote{Options overlay can add 5-15\% annual yield in high-vol environments.}
\end{frame}

\begin{frame}[t]{Covered Call Strategy}
\begin{center}
\includegraphics[width=0.65\textwidth]{../charts/08a_covered_call_payoff/chart.pdf}
\end{center}
\bottomnote{Trade unlimited upside for immediate premium income.}
\end{frame}

\begin{frame}[t]{Covered Call Mathematics}
\textbf{Position Composition}
\begin{itemize}
    \item Long: 1 SOL at \$150
    \item Short: 1 Call, Strike \$180, Premium \$15
\end{itemize}

\textbf{Payoff at Expiration}
\begin{equation}
    \Pi = \min(K, S_T) - S_0 + C
\end{equation}

\textbf{Key Metrics}
\begin{itemize}
    \item Maximum profit: $(K - S_0) + C = \$30 + \$15 = \$45$ (30\%)
    \item Breakeven: $S_0 - C = \$150 - \$15 = \$135$
    \item Premium yield: $C/S_0 = 15/150 = 10\%$ per period
\end{itemize}
\bottomnote{Repeat monthly for 20-40\% annualized premium income in high-vol periods.}
\end{frame}

\begin{frame}[t]{Cash-Secured Put for Accumulation}
\textbf{Strategy Logic}
\begin{itemize}
    \item Sell put at desired entry price (e.g., \$120)
    \item Receive premium immediately
    \item If assigned: Buy SOL at target price (win)
    \item If not assigned: Keep premium (win)
\end{itemize}
\vspace{0.3cm}
\textbf{Example}
\begin{itemize}
    \item Sell Put: Strike \$120, Premium \$8
    \item If SOL drops to \$100: Buy at \$120, effective cost \$112
    \item If SOL stays above \$120: Keep \$8 premium (6.7\% yield)
\end{itemize}
\bottomnote{Systematic put selling accumulates SOL at progressively lower prices.}
\end{frame}

%=============================================================================
\section{Monte Carlo Simulations}
%=============================================================================

\begin{frame}[t]{Price Path Modeling}
\textbf{Geometric Brownian Motion with Jumps (Merton)}
\begin{equation}
    dS_t = (\mu - \lambda\kappa)S_t dt + \sigma S_t dW_t + S_t dJ_t
\end{equation}

\textbf{Parameters for SOL}
\begin{itemize}
    \item $\mu = 15\%$: Expected annual return (includes staking)
    \item $\sigma = 80\%$: Annualized volatility
    \item $\lambda = 2$: Jump intensity (2 events/year avg)
    \item $\mu_J = -15\%$: Mean jump size (crashes)
    \item $\sigma_J = 10\%$: Jump size volatility
\end{itemize}
\bottomnote{Jump component captures crypto market crash dynamics.}
\end{frame}

\begin{frame}[t]{Monte Carlo Fan Chart}
\begin{center}
\includegraphics[width=0.65\textwidth]{../charts/07_monte_carlo_fan/chart.pdf}
\end{center}
\bottomnote{5,000 simulated paths over 3-year horizon show distribution of outcomes.}
\end{frame}

\begin{frame}[t]{Accumulation Strategy Comparison}
\begin{center}
\includegraphics[width=0.65\textwidth]{../charts/09_dca_vs_lumpsum/chart.pdf}
\end{center}
\bottomnote{Lump sum has higher expected value but DCA reduces timing risk.}
\end{frame}

\begin{frame}[t]{Risk Metrics}
\textbf{Value at Risk (VaR)}
\begin{equation}
    VaR_\alpha = -\mu T + \sigma\sqrt{T} \cdot \Phi^{-1}(\alpha)
\end{equation}

\textbf{Conditional VaR (Expected Shortfall)}
\begin{equation}
    CVaR_\alpha = -\mu T + \sigma\sqrt{T} \cdot \frac{\phi(\Phi^{-1}(\alpha))}{1-\alpha}
\end{equation}

\textbf{Example}: \$100K portfolio, 3-year horizon, 95\% confidence
\begin{itemize}
    \item VaR (95\%): 35\% loss = \$35K maximum expected loss
    \item CVaR (95\%): 48\% loss = \$48K expected loss if VaR breached
    \item Note: Staking compounds during drawdowns, reducing realized losses
\end{itemize}
\bottomnote{Staking yield provides 25\% additional SOL over 3 years, offsetting losses.}
\end{frame}

%=============================================================================
\section{Opportunistic Acquisition Framework}
%=============================================================================

\begin{frame}[t]{When to Deploy Capital}
\textbf{Technical Signals}
\begin{itemize}
    \item RSI below 30: Oversold condition
    \item Price below 200-day MA: Long-term support
    \item Volume spike with price decline: Capitulation
\end{itemize}
\vspace{0.3cm}
\textbf{Fundamental Metrics}
\begin{itemize}
    \item Staking ratio increasing: Supply squeeze
    \item TVL growth in DeFi: Ecosystem health
    \item Developer activity: GitHub commits, new projects
\end{itemize}
\vspace{0.3cm}
\textbf{Macro Factors}
\begin{itemize}
    \item Fed policy: Risk-on during dovish periods
    \item BTC correlation: SOL typically follows BTC cycles
    \item Regulatory news: SEC clarity as catalyst
\end{itemize}
\bottomnote{Combine technical, fundamental, and macro signals for entry timing.}
\end{frame}

\begin{frame}[t]{DCA vs Opportunistic Buying}
\textbf{Dollar Cost Averaging (DCA)}
\begin{itemize}
    \item Fixed schedule: \$X every month
    \item Removes timing decisions
    \item Works well in sideways/volatile markets
\end{itemize}

\textbf{Value Averaging (VA)}
\begin{itemize}
    \item Target portfolio growth rate
    \item Buy more when down, less when up
    \item Requires cash reserves for large buys
\end{itemize}

\textbf{Opportunistic (Signal-Based)}
\begin{itemize}
    \item Deploy capital at triggered signals
    \item Requires discipline and cash reserves
    \item Higher expected return, higher variance
\end{itemize}
\bottomnote{Hybrid approach: DCA base + opportunistic overlay during extreme events.}
\end{frame}

%=============================================================================
\section{Hybrid Model: Validator + Corporate Finance}
%=============================================================================

\begin{frame}[t]{Integrated Strategy}
\begin{center}
\includegraphics[width=0.60\textwidth]{../charts/11_hybrid_model_flow/chart.pdf}
\end{center}
\bottomnote{Combine validator operations with corporate capital structure for maximum yield.}
\end{frame}

\begin{frame}[t]{Revenue Stacking}
\textbf{Layer 1: Staking Yield}
\begin{itemize}
    \item Base APY: 7.5\% on all SOL holdings
    \item Validator self-stake: Full yield retained
\end{itemize}

\textbf{Layer 2: Validator Commission}
\begin{itemize}
    \item 5\% commission on delegated stake
    \item Additional 0.5-1\% effective yield
\end{itemize}

\textbf{Layer 3: MEV Revenue}
\begin{itemize}
    \item Jito tips: 0.2-0.5\% additional yield
    \item Scales with stake weight
\end{itemize}

\textbf{Layer 4: Options Overlay}
\begin{itemize}
    \item Covered calls: 5-15\% additional
    \item Risk: Caps upside in rallies
\end{itemize}
\bottomnote{Combined yield potential: 15-25\% annual before price appreciation.}
\end{frame}

\begin{frame}[t]{IRR Sensitivity Analysis}
\begin{center}
\includegraphics[width=0.65\textwidth]{../charts/12_irr_sensitivity/chart.pdf}
\end{center}
\bottomnote{Leverage amplifies returns in both directions; maintain interest coverage.}
\end{frame}

\begin{frame}[t]{Implementation Roadmap}
\textbf{Phase 1: Foundation (Months 1-3)}
\begin{itemize}
    \item Establish legal entity structure
    \item Initial equity raise
    \item Begin SOL accumulation via DCA
\end{itemize}

\textbf{Phase 2: Validator Setup (Months 4-6)}
\begin{itemize}
    \item Deploy validator infrastructure
    \item Stake initial SOL holdings
    \item Enable Jito MEV participation
\end{itemize}

\textbf{Phase 3: Capital Structure (Months 7-12)}
\begin{itemize}
    \item Issue convertible bonds (if public)
    \item Execute PIPE placements
    \item Begin options overlay strategy
\end{itemize}
\bottomnote{Conservative ramp-up allows operational learning and risk management.}
\end{frame}

%=============================================================================
\section{Conclusion \& Interactive Tools}
%=============================================================================

\begin{frame}[t]{Key Takeaways}
\begin{enumerate}
    \item \textbf{SOL offers unique yield}: 7-9\% staking APY unavailable with BTC
    \item \textbf{Validator business adds margin}: Commission + MEV revenue
    \item \textbf{Corporate finance amplifies}: Convertibles provide cheap capital
    \item \textbf{Capital structure matters}: Optimize leverage for WACC minimization
    \item \textbf{Options enhance yield}: 5-15\% additional via covered calls
    \item \textbf{Risk management critical}: Monte Carlo for scenario planning
    \item \textbf{Hybrid model maximizes}: Stack multiple yield sources
\end{enumerate}
\bottomnote{Target: 15-25\% annual ROE through integrated strategy.}
\end{frame}

\begin{frame}[t]{Interactive Calculators}
\textbf{Available at:} \url{https://digital-ai-finance.github.io/solana-staking}

\begin{itemize}
    \item \textbf{Validator Economics}: Profitability by stake and commission
    \item \textbf{Convertible Analyzer}: Bond valuation and Greeks
    \item \textbf{Options Strategy Builder}: Payoff diagrams and premium
    \item \textbf{Monte Carlo Simulator}: Distribution of outcomes
    \item \textbf{Capital Structure Optimizer}: WACC and leverage analysis
\end{itemize}
\vspace{0.5cm}
\begin{center}
\textit{All tools run client-side -- no data sent to servers}
\end{center}
\bottomnote{Source code available on GitHub for transparency and audit.}
\end{frame}

\begin{frame}[t]{Mathematical Appendix}
\textbf{Full derivations available in whitepaper:}
\begin{itemize}
    \item Merton jump-diffusion derivation
    \item Black-Scholes from first principles
    \item MM capital structure optimization
    \item VaR and CVaR proofs
    \item Validator economics model
    \item Options strategy payoff derivations
\end{itemize}
\vspace{0.5cm}
\textbf{GitHub Repository}
\begin{itemize}
    \item Python models: \texttt{models/*.py}
    \item Chart generation: \texttt{charts/*/chart.py}
    \item JavaScript calculators: \texttt{docs/calculators/}
\end{itemize}
\bottomnote{All code open source under MIT license.}
\end{frame}

\begin{frame}
\begin{center}
\Huge\textcolor{solpurple}{Questions?}

\vspace{1cm}
\large
\texttt{github.com/Digital-AI-Finance/solana-staking}

\vspace{0.5cm}
\normalsize
Slides, models, and calculators all open source
\end{center}
\end{frame}

\end{document}
